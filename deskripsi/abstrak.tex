% Mengubah keterangan `Abstract` ke bahasa indonesia.
% Hapus bagian ini untuk mengembalikan ke format awal.
\renewcommand\abstractname{Abstrak}

\begin{abstract}

  % Ubah paragraf berikut sesuai dengan abstrak dari penelitian.
  Pandemi COVID-19 telah melihat gerakan internasional menuju pengajaran dan penilaian online. Perpindahan secara online sering kali diselesaikan dalam waktu singkat dan dengan sedikit peluang untuk membuat rencana guna memastikan integritas akademik tetap terjaga. Sifat daripada penilaian yang sedemikian rupa sehingga, apakah itu diadakan secara langsung atau online, siswa memiliki insentif pribadi untuk mencoba dan mendapatkan nilai terbaik yang mereka bisa. Integritas akademik didefinisikan sebagai komitmen terhadap nilai-nilai moral mendasar seperti kejujuran, kepercayaan, keadilan, kesopanan, rasa hormat, dan tanggung jawab. Kekhawatiran tentang isu kecurangan ujian online dalam konteks COVID-19 juga tercermin di media-media di Spanyol, yang menggemakan banyak kasus kecurangan dalam penilaian online selama tahun 2020. Pada literatur-literatur penelitian digambarkan sejumlah besar perilaku yang berhubungan dengan perilaku akademik yang tidak pantas dalam lingkungan pembelajaran tradisional non-online. Ketidakjujuran akademik telah menyibukkan akademisi selama bertahun-tahun, tetapi fenomena ini telah meningkat dalam beberapa tahun terakhir. Meskipun pengawasan online memberikan beberapa keuntungan intuitif untuk mendeteksi perilaku curang, dan ini memetakan secara dekat ke proses pengawasan tatap muka yang sudah dikenal, banyak yang telah mengangkat kekhawatiran di media tentang etika dan kemanjuran sistem ini. Siswa direkam video menggunakan webcam mereka sendiri selama ujian. Jika acara curang terdeteksi, program akan menandai video untuk dilihat di masa mendatang oleh pengawas. Dengan demikian, tuntutan waktu pengawas berkurang, namun siswa dipantau. Program peringkasan video dapat menghasilkan bingkai utama (kumpulan gambar yang diekstrak dari sumber video) atau skim video (segmen video yang diekstrak dari sumber video) untuk menunjukkan potensi perilaku curang. Sehubungan dengan ujian online, perekaman video web mengacu pada situasi di mana siswa merekam video selama keseluruhan ujian untuk dilihat nanti oleh instruktur. Penggunaan biometrik, pengukuran fitur fisiologis atau perilaku individu, adalah metode otentikasi yang memungkinkan verifikasi identitas berkelanjutan. Saat menggunakan penilaian yang memerlukan jawaban tertulis, perangkat lunak yang memeriksa keaslian teks (seperti "TurnItIn") dapat membantu mengidentifikasi karya yang diambil dari sumber tanpa kutipan yang tepat. Dengan metode ini, karya yang dikirimkan dibandingkan dengan karya lain yang disimpan di bank perangkat lunak untuk memeriksa keasliannya. Penguncian browser mencegah penggunaan materi elektronik tambahan selama ujian dengan memblokir siswa dari mengunjungi situs web eksternal atau menggunakan aplikasi yang tidak sah pada perangkat yang sama dengan yang digunakan untuk mengikuti penilaian.

\end{abstract}

% Mengubah keterangan `Index terms` ke bahasa indonesia.
% Hapus bagian ini untuk mengembalikan ke format awal.
\renewcommand\IEEEkeywordsname{Kata kunci}

\begin{IEEEkeywords}

  % Ubah kata-kata berikut sesuai dengan kata kunci dari penelitian.
  Integritas, Kecurangan, Pendidikan, Akademik, Evaluasi, Biometrik.

\end{IEEEkeywords}
