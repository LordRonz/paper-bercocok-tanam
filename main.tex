% Harus dimuat terlebih dahulu, digunakan agar file PDF memiliki format karakter yang benar.
% Untuk informasi lebih lanjut, lihat https://ctan.org/pkg/cmap.
\RequirePackage{cmap}

% Format dokumen sebagai paper konferensi menggunakan aturan IEEEtran terbaru (v1.8b).
% Untuk informasi lebih lanjut, lihat http://www.michaelshell.org/tex/ieeetran/.
\documentclass[conference]{IEEEtran}[2015/08/26]

% Format encoding font dan input menjadi 8-bit UTF-8.
\usepackage[T1]{fontenc}
\usepackage[utf8]{inputenc}

% Format bahasa menjadi bahasa german dan inggris.
\usepackage[indonesian]{babel}

% Digunakan untuk tujuan demonstrasi.
\usepackage{mwe}

% Digunakan untuk menampilkan font dengan style yang lebih baik.
\usepackage[zerostyle=b,scaled=.75]{newtxtt}

% Digunakan untuk menampilkan tabel dengan style yang lebih baik.
\usepackage{booktabs}

% Digunakan untuk menampilkan gambar pada dokumen.
\usepackage{graphicx}

% Digunakan untuk menampilkan potongan kode.
\usepackage{listings}
\lstset{
  basicstyle=\ttfamily,
  columns=fixed,
  basewidth=.5em,
  xleftmargin=0.5cm,
  captionpos=b
}

% Digunakan agar backticks (`) dapat dirender pada PDF.
% Untuk informasi lebih lanjut, lihat https://tex.stackexchange.com/a/341057/9075.
\usepackage{upquote}

% Digunakan untuk menyeimbangkan bagian akhir dokumen dengan dua kolom.
\usepackage{balance}

\usepackage[table]{xcolor}

% Digunakan untuk menampilkan pustaka.
\usepackage[square,comma,numbers,sort&compress]{natbib}

% Mengubah format ukuran teks pada natbib.
\renewcommand{\bibfont}{\normalfont\footnotesize}

% Menambah nama penulis ketika menggunakan perintah \citet.
% Untuk informasi lebih lanjut, lihat https://tex.stackexchange.com/a/76075/9075.
\usepackage{etoolbox}
\makeatletter
\patchcmd{\NAT@test}{\else \NAT@nm}{\else \NAT@hyper@{\NAT@nm}}{}{}
\makeatother

\usepackage[hyphens]{url}

% Digunakan untuk menambah hyperlink pada referensi.
\usepackage{hyperref}

% Menonaktifkan warna dan bookmark pada hyperref.
\hypersetup{hidelinks,
  colorlinks=true,
  allcolors=black,
  pdfstartview=Fit,
  breaklinks=true
}

% Digunakan untuk membenarkan hyperref pada gambar.
\usepackage[all]{hypcap}

% Digunakan untuk menampilkan beberapa gambar
\usepackage[caption=false,font=footnotesize]{subfig}

\usepackage{stfloats}

% Tambahkan format tanda hubung yang benar di sini
\hyphenation{
  ro-ket
  me-ngem-bang-kan
  per-hi-tu-ngan
}

\begin{document}

  % Ubah kalimat berikut sesuai dengan judul penelitian.
\title{Integritas Akademik dan Analisis Kecurangan\\Pada Pembelajaran Daring Serta Penanggulangannya}

% Ubah kalimat-kalimat berikut sesuai dengan nama, institusi, alamat dan kontak penulis.
\author{
  \IEEEauthorblockN{Aaron Christopher Tanhar}
  \IEEEauthorblockA{Departemen Teknik Komputer\\
    FTEIC\\
    Institut Teknologi Sepuluh Nopember\\
    Surabaya, Indonesia 60111}
  
  \and
  \IEEEauthorblockN{Dion Andreas Solang}
  \IEEEauthorblockA{Departemen Teknik Komputer\\
    FTEIC\\
    Institut Teknologi Sepuluh Nopember\\
    Surabaya, Indonesia 60111}

  \and
  \IEEEauthorblockN{Alfredo Gerald Hartono}
  \IEEEauthorblockA{Departemen Teknik Komputer\\
    FTEIC\\
    Institut Teknologi Sepuluh Nopember\\
    Surabaya, Indonesia 60111}
}

% Digunakan untuk menampilkan judul dan deskripsi penulis.
\maketitle

  % Mengubah keterangan `Abstract` ke bahasa indonesia.
% Hapus bagian ini untuk mengembalikan ke format awal.
\renewcommand\abstractname{Abstrak}

\begin{abstract}

  % Ubah paragraf berikut sesuai dengan abstrak dari penelitian.
  Pandemi COVID-19 telah melihat gerakan internasional menuju pengajaran dan penilaian online. Perpindahan secara online sering kali diselesaikan dalam waktu singkat dan dengan sedikit peluang untuk membuat rencana guna memastikan integritas akademik tetap terjaga. Sifat daripada penilaian yang sedemikian rupa sehingga, apakah itu diadakan secara langsung atau online, siswa memiliki insentif pribadi untuk mencoba dan mendapatkan nilai terbaik yang mereka bisa. Integritas akademik didefinisikan sebagai komitmen terhadap nilai-nilai moral mendasar seperti kejujuran, kepercayaan, keadilan, kesopanan, rasa hormat, dan tanggung jawab. Kekhawatiran tentang isu kecurangan ujian online dalam konteks COVID-19 juga tercermin di media-media di Spanyol, yang menggemakan banyak kasus kecurangan dalam penilaian online selama tahun 2020. Pada literatur-literatur penelitian digambarkan sejumlah besar perilaku yang berhubungan dengan perilaku akademik yang tidak pantas dalam lingkungan pembelajaran tradisional non-online. Ketidakjujuran akademik telah menyibukkan akademisi selama bertahun-tahun, tetapi fenomena ini telah meningkat dalam beberapa tahun terakhir. Meskipun pengawasan online memberikan beberapa keuntungan intuitif untuk mendeteksi perilaku curang, dan ini memetakan secara dekat ke proses pengawasan tatap muka yang sudah dikenal, banyak yang telah mengangkat kekhawatiran di media tentang etika dan kemanjuran sistem ini. Siswa direkam video menggunakan webcam mereka sendiri selama ujian. Jika acara curang terdeteksi, program akan menandai video untuk dilihat di masa mendatang oleh pengawas. Dengan demikian, tuntutan waktu pengawas berkurang, namun siswa dipantau. Program peringkasan video dapat menghasilkan bingkai utama (kumpulan gambar yang diekstrak dari sumber video) atau skim video (segmen video yang diekstrak dari sumber video) untuk menunjukkan potensi perilaku curang. Sehubungan dengan ujian online, perekaman video web mengacu pada situasi di mana siswa merekam video selama keseluruhan ujian untuk dilihat nanti oleh instruktur. Penggunaan biometrik, pengukuran fitur fisiologis atau perilaku individu, adalah metode otentikasi yang memungkinkan verifikasi identitas berkelanjutan. Saat menggunakan penilaian yang memerlukan jawaban tertulis, perangkat lunak yang memeriksa keaslian teks (seperti "TurnItIn") dapat membantu mengidentifikasi karya yang diambil dari sumber tanpa kutipan yang tepat. Dengan metode ini, karya yang dikirimkan dibandingkan dengan karya lain yang disimpan di bank perangkat lunak untuk memeriksa keasliannya. Penguncian browser mencegah penggunaan materi elektronik tambahan selama ujian dengan memblokir siswa dari mengunjungi situs web eksternal atau menggunakan aplikasi yang tidak sah pada perangkat yang sama dengan yang digunakan untuk mengikuti penilaian.

\end{abstract}

% Mengubah keterangan `Index terms` ke bahasa indonesia.
% Hapus bagian ini untuk mengembalikan ke format awal.
\renewcommand\IEEEkeywordsname{Kata kunci}

\begin{IEEEkeywords}

  % Ubah kata-kata berikut sesuai dengan kata kunci dari penelitian.
  Integritas, Kecurangan, Pendidikan, Akademik, Evaluasi, Biometrik.

\end{IEEEkeywords}


  % Ubah bagian berikut sesuai dengan konten-konten yang akan dimasukkan pada dokumen
  % Ubah judul dan label berikut sesuai dengan yang diinginkan.
\section{Pendahuluan}
\label{sec:pendahuluan}

% Ubah paragraf-paragraf pada bagian ini sesuai dengan yang diinginkan.

Di bidang teknik elektro, sebuah switch adalah komponen elektronik yang dapat menghubungkan maupun memutuskan jalur konduksi dari sebuah rangkaian listrik, memutus arus listrik atau mengalihkannya dari satu konduktor ke konduktor lainnya. Sederhananya, sebuah switch dapat membuat atau memutuskan rangkaian listrik. Setiap aplikasi listrik dan elektronik menggunakan setidaknya satu switch atau sakelar untuk melakukan operasi ON dan OFF perangkat.

Maka dapat diketahui bahwa switch merupakan bagian dari sistem kontrol, dan tanpa switch, kita tidak dapat memiliki sistem kontrol. Sebuah switch dapat melakukan dua fungsi, yakni ON dengan menutup kontaknya, ataupun OFF dengan membuka kontaknya.

Tipe paling umum dari sebuah switch adalah sebuah perangkat elektromekanikal yang terdiri dari satu atau lebih set kontak listrik yang bergerak yang terhubung pada rangkaian eksternal. Ketika sepasang kontak bersentuhan, arus dapat mengalir di antara kontak tersebut, sedangkan ketika kontak dipisahkan tidak ada arus yang dapat mengalir.

Aktuator adalah komponen dari sebuah mesin atau alat yang berfungsi untuk menggerakkan dan mengendalikan suatu mekanisme atau sistem, misalnya saja dengan membuka sebuah valve. Secara sederhana, aktuator adalah sebuah "penggerak".

Aktuator membutuhkan control signal dan sumber energi. Sinyal kontrol adalah energi yang relatif rendah dan mungkin saja berupa tegangan atau arus listrik, pneumatik, ataupun berupa tekanan cairan hidrolik, atau bahkan tenaga manusia. Sumber energi utama dari aktuator dapat berupa arus listrik, tekanan hidrolik, atau tekanan pneumatik. Ketika menerima sinyal kontrol, aktuator meresponnya dengan mengubah energi dari sumber menjadi gerakan mekanis. Dalam pengertian listrik, hidrolik, dan pneumatik, ini adalah bentuk otomatisasi atau sistem kontrol otomatis.

  \section{Integritas Akademik}
\label{sec:integritasakademik}

Integritas akademik didefinisikan sebagai komitmen terhadap nilai-nilai moral mendasar seperti kejujuran, kepercayaan, keadilan, kesopanan, rasa hormat, dan tanggung jawab \citet{keohane1999fundamental}. Integritas akademik didefinisikan sebagai komitmen terhadap enam nilai inti, yaitu kejujuran, kepercayaan, keadilan, rasa hormat, tanggung jawab, dan keberanian, dalam semua aspek praktik ilmiah, bahkan dalam menghadapi kesulitan. Nilai-nilai ini penting di lembaga pendidikan tinggi untuk evaluasi pembelajaran, tetapi juga karena lembaga ini diharapkan memungkinkan dan mendorong perolehan pengetahuan, pembelajaran individu, pengembangan kemampuan intelektual, pengembangan otonomi dan pemeliharaan reputasi keunggulan akademik sekolah \citep{ahmed2018student} \citep{nuss1984academic}, serta menghasilkan lulusan yang berkontribusi bagi pembangunan ekonomi, sosial dan kemanusiaan negara \citep{muhammad2020factors}, dan yang berperilaku bermoral dalam masyarakat. Dengan demikian, tujuan utama belajar mengajar seperti yang dirasakan saat ini adalah untuk menumbuhkan lingkungan yang berorientasi pada pembelajaran, berdasarkan motivasi pribadi, lebih dari menciptakan lingkungan yang berorientasi pada prestasi \citep{bertram2017academic}. Dan memang, ketika siswa belajar melalui motivasi intrinsik, praktik akademik biasanya adil \citep{barbaranelli2018machiavellian} \citep{krou2020achievement}. Nilai-nilai ini, yang mendasari integritas akademik, dianggap penting bahkan di saat-saat tertekan, yang mungkin berasal dari kurangnya pengetahuan dan ketakutan akan kegagalan \citep{keohane1999fundamental}, atau dari krisis seperti pandemi saat ini \citep{moralista2020faculty}. 

Studi tentang integritas akademik di Spanyol, cakupan geografis makalah ini, tidak memiliki tradisi yang mapan seperti lingkungan Anglo-Saxon atau Eropa tengah dan utara \citep{forgas2009ciberplagio}. Mengenai studi yang berfokus pada perilaku siswa yang tidak jujur terlihat pada saat mereka mengikuti ujian, pekerjaan yang dapat dikutip sangat langka.

Pada sebuah studi yang difokuskan pada mahasiswa keperawatan \citep{blanch2006nivel} menunjukkan bahwa 28\% siswa mengaku telah menyalin selama ujian. Data dari studi kedua, berdasarkan sampel mahasiswa universitas Spanyol, menunjukkan bahwa sekitar 45\% mahasiswa mengaku telah menggunakan lembar contekan dan materi yang tidak diperbolehkan selama ujian \citep{sureda2009practicas}. Sebuah studi kemudian dilakukan oleh kelompok peneliti yang sama, menemukan bahwa hampir 50\% mahasiswa melaporkan telah menyalin setidaknya sekali selama ujian tatap muka \citep{comas2011integridad}. Sebuah panel ahli Spanyol, menyatakan bahwa perilaku tidak jujur paling serius yang dapat dilakukan mahasiswa dalam penilaian mereka adalah: meniru identitas orang lain dalam penilaian; mencuri tes atau ujian, memanipulasi nilai mereka dan mengubahnya untuk orang lain; memperoleh soal ujian atau penilaian sebelum mengikuti ujian; menyerahkan ujian yang diambil oleh siswa lain sebagai ujiannya sendiri; menyontek pada ujian tatap muka melalui perangkat teknologi, seperti ponsel dan earpiece, kemudian, akhirnya, menghadirkan karya siswa lain sebagai miliknya \citep{sureda2020conductas}. Hubungan dan penilaian perilaku tidak jujur menunjukkan kekhawatiran tentang penipuan ujian oleh kelompok ahli yang berpartisipasi.

Adaptasi sistem pendidikan universitas Spanyol dengan konteks yang disebabkan oleh pandemi telah menyebabkan, antara lain, peningkatan kekhawatiran tentang penipuan ujian. Kekhawatiran tersebut telah menghasilkan pengembangan pedoman dan rekomendasi oleh otoritas politik dan akademik tentang prosedur penilaian non-tatap muka \citep{castells2020recomendaciones} \citep{cordon2020informe} \citep{gonzalez2020informe}. Dalam pedoman Konferensi Rektor Universitas Spanyol (Conferencia de Rectores de las Universidades Españolas - CRUE), tidak ada referensi eksplisit untuk menyontek dalam ujian, tetapi hingga dua puluh kali, kata "keamanan" muncul, dengan kejujuran menjadi salah satu dimensi fundamentalnya. Secara khusus, berikut ini dinyatakan:
Aspek penting lainnya yang perlu dipertimbangkan adalah langkah-langkah untuk menjaga integritas akademik dan penggunaan mekanisme hukum yang tersedia (pengusiran dari tes, kualifikasi penangguhan atau, jika perlu, lembaga proses disipliner) dalam kasus tes atau tugas yang curang \citep{cordon2020informe}.

Dalam sebuah buku pegangan yang disiapkan oleh Ministerio de Universidades (Kementerian Universitas), satu bagian didedikasikan untuk menyajikan rekomendasi untuk menghindari penggunaan cara-cara curang dan satu bagian lagi untuk menyajikan sistem untuk menjamin kepenulisan ujian \citet{gonzalez2020informe}.

Buku pegangan rekomendasi yang dikembangkan oleh Kelompok Otoritas Pengajaran Daring Universitas Negeri Castilla y León patut dicatat \citep{garciaevaluacion}. Di antara rekomendasinya adalah mendeteksi peniruan identitas selama ujian sebagai persyaratan yang dapat diminta dari sistem e-proctoring, memblokir browser peserta ujian sehingga mereka tidak dapat mengakses konten di luar ujian, mendeteksi elemen selain yang diperlukan untuk melakukan pengujian; dan, terakhir, mendorong diperolehnya bukti objektif tentang penyelesaian ujian oleh siswa tanpa bantuan atau kerjasama dari pihak ketiga.

Kekhawatiran tentang isu kecurangan ujian online dalam konteks COVID-19 juga tercermin di media-media di Spanyol, yang menggemakan banyak kasus kecurangan dalam penilaian online selama tahun 2020. Di sebagian besar artikel jurnalistik ini, penilaian online disajikan dengan sudut pandang negatif karena potensi kemudahan penipuan. Menurut pendapat Goberna, Profesor Matematika di Universitas Alicante, “Ujian online adalah penipuan; mereka pada dasarnya akan menipu”. Seorang Profesor Filologi Italia di Universitas Oviedo, de Sande, mempertahankan posisi serupa yang menyatakan "Dengan ujian telematika, Anda memberikan kursus".

Indikator akhir sejauh mana fenomena kecurangan penilaian dalam konteks COVID-19 dapat diperoleh dengan menelusuri YouTube dengan deskriptor “copiar examen online” (kecurangan ujian online”). Banyak ditemukan video di mana pengalaman menyontek saat ujian terkait dengan judul langsung, seperti “Ayudo a mi hermana a copiar en un examen online!” (“Saya membantu adik saya menyontek dalam ujian online!”), yang memperoleh lebih dari 3,7 juta tampilan dalam waktu kurang dari sembilan bulan. dan lainnya secara terbuka memberikan saran tentang menyontek pada penilaian online, seperti "Cómo saber las respuestas de un examen online" ("Cara mengetahui jawaban ujian online"), yang mengumpulkan hampir 850.000 tampilan dari April 2020 hingga Februari 2021.

Mempertimbangkan hal-hal yang sudah dirincikan di atas, relevansi memperoleh pengetahuan baru tentang kecurangan ujian di era COVID-19 menjadi jelas. Studi ini membahas masalah dari perspektif yang jarang digunakan sampai sekarang, yaitu analisis data dari pencarian Internet atau analisis pencarian.
  \section{Ketidakjujuran Akademik}
\label{sec:ketidakjujuranakademik}

Berkebalikan dengan integritas akademik, ketidakjujuran akademik didefinisikan sebagai perilaku tidak etis dalam lingkungan akademik \citep{muhammad2020factors}. Ini adalah perilaku yang tidak pantas dimana siswa bertindak untuk mendapatkan keuntungan akademik yang tidak adil untuk diri mereka sendiri atau untuk teman-teman mereka di komunitas akademik \citep{grira2019rationality}. Ketidakjujuran akademik mencegah perkembangan nilai-nilai positif seperti kejujuran, keadilan dan kemajuan belajar yang signifikan, dan terkait dengan perilaku negatif lainnya, yang memiliki implikasi bahkan di luar akademik \citet{krou2020achievement} \citep{yu2018college}, seperti di pasar kerja di mana lulusan dengan keterampilan yang tidak tepat dapat dipekerjakan \citep{barbaranelli2018machiavellian} \citep{bashir2018development}.

Penelitian menunjukkan bahwa perilaku seperti itu adalah fenomena yang diketahui dan lazim yang telah meningkat selama beberapa tahun terakhir \citep{birks2020managing} \citep{grira2019rationality} \citep{harper2021detecting}, dan juga bahwa ini adalah fenomena global lintas budaya yang memiliki banyak segi. Misalnya, penelitian di India menemukan bahwa sedikit lebih dari 20\% dari 1.369 peserta penelitian mengakui ketidakjujuran akademis \citep{stearns2001student}. Demikian pula, salah satu studi terluas dan terlama yang dilakukan di Australia memeriksa 150.000 siswa selama delapan tahun dan menemukan bahwa 65\% siswa melaporkan ketidakjujuran akademik dalam setidaknya satu parameter penelitian \citep{duff2006international}. Demikian pula, penelitian yang dilakukan di Rumania menemukan bahwa 95\% siswa melaporkan perilaku akademik yang tidak pantas \citep{ives2017patterns}.

Pada literatur-literatur penelitian digambarkan sejumlah besar perilaku yang berhubungan dengan perilaku akademik yang tidak pantas dalam lingkungan pembelajaran tradisional non-online, termasuk: membantu teman dalam ujian, bekerja sama dengan teman sebaya selama ujian, penggunaan larangan bahan ujian, penggunaan bahan teman, mengizinkan pekerjaan untuk disalin, mendapatkan solusi dari teman yang telah mengikuti ujian, mengikuti ujian untuk orang lain, plagiarisme (termasuk materi yang disalin tanpa memberikan kredit kepada penulis, penggunaan berulang tugas yang sudah diserahkan, karya yang ditulis oleh pihak ketiga dan disajikan sebagai karya siswa atau membeli karya – menyontek kontrak), kerjasama antar teman untuk menulis karya ketika tidak ada izin untuk melakukannya dan menambahkan sumber ke daftar pustaka tanpa menggunakannya \citep{denisova2017challenges} \citep{harper2021detecting} \citep{von2001can} \citep{yu2018college}.  

Sebuah studi baru-baru ini melaporkan bahwa sebagian besar perilaku yang dianggap kurang integritas akademik terkait dengan bantuan selama ujian, yang paling umum adalah memberi dan menerima bantuan teman dalam ujian pilihan ganda dan dalam ujian di mana jawaban singkat diperlukan \citep{harper2021detecting}.

Selain itu, terdapat suatu peningkatan motivasi siswa untuk membayar faktor luar untuk mengerjakan tugas (kecurangan kontrak) \citep{birks2020managing}. Ditemukan bahwa alasan utama penggunaan kecurangan kontrak adalah ketidakpuasan dengan lingkungan belajar mengajar, kurangnya waktu dan persepsi bahwa ada banyak peluang untuk menyontek \citep{bretag2019contract} \citep{foltynek2018analysis}. Di Australia, ditemukan bahwa siswa menggunakan kecurangan kontrak yang disediakan oleh lingkaran sosial terdekat mereka daripada kontraktor eksternal \citep{harper2021detecting}.

Sebuah meta-analisis yang baru-baru ini diterbitkan oleh \citet{krou2020achievement} dari berbagai studi penelitian yang menyelidiki antara lain, perilaku di berbagai bidang (sains, teknologi, teknik, matematika, dan aliran bisnis) mengkategorikan perilaku terkait ketidakjujuran akademik menjadi dua kategori: plagiarisme (materi disalin tanpa memberikan kredit kepada penulis, karya yang ditulis oleh pihak ketiga dan dipresentasikan sebagai hasil karya siswa dan lain-lain) dan mencontek (menerima jawaban dari siswa yang sudah menyelesaikan ujian, mengerjakan tugas bersama teman tanpa izin, menyalin saat ujian, dan menggunakan bahan bantu tanpa izin selama ujian sedang berlangsung). Selain itu, ditemukan bahwa siswa yang menyaksikan perilaku akademik yang tidak pantas dari temannya cenderung melakukan perilaku serupa, berbeda dengan siswa yang tidak menyaksikan perilaku tersebut \citep{ahmed2018student} \citep{barbaranelli2018machiavellian} \citep{kiekkas2020reasons}. Dengan kata lain, norma perilaku akademik yang pantas atau tidak pantas mempengaruhi perilaku siswa. Penyebab ketidakjujuran akademik siswa banyak dan beragam dan bersumber dari faktor pribadi-intrinsik atau ekstrinsik. Faktor intrinsik pribadi meliputi motivasi yang kuat untuk berhasil, daya saing, takut gagal, pengetahuan yang tidak memadai dalam disiplin, rasa efikasi diri yang berkurang, studi yang berlebihan, disiplin diri yang tidak memadai, kemalasan, kelelahan, kecenderungan impulsif, sebelumnya. prestasi akademik yang rendah dan perkembangan moral yang rendah. Faktor ekstrinsik termasuk pengabaian perilaku tidak etis oleh anggota staf dan tidak adanya implikasi disiplin untuk menyontek, tekanan orang tua untuk berhasil, ketidakpuasan dengan pengajaran, perasaan bahwa ada banyak peluang menyontek, tekanan waktu untuk menyerahkan tugas, akademik yang terlalu tinggi. tuntutan, konten yang tidak relevan dengan profesi masa depan siswa, keinginan untuk mencapai status sosial yang lebih baik dan keinginan untuk memasuki pasar kerja \citep{amigud2019246} \citep{birks2020managing} \citep{bretag2019contract} \citep{kiekkas2020reasons} \citep{krou2020achievement} \citep{murdock2006motivational}.

Alasan perilaku ini dapat dikategorikan menurut tiga mekanisme motivasi: (1) Apa tujuan saya? Ini termasuk pertimbangan motivasi intrinsik dan ekstrinsik siswa; (2) Bisakah saya melakukan ini? Ini termasuk motivasi ekstrinsik siswa, efikasi diri dan lingkungan belajar mereka, termasuk ketidakmampuan belajar, ujian yang tidak jelas, dan keinginan untuk menjadi seperti peserta didik lainnya \citep{bertram2017academic} \citep{etgar2019white} \citep{murdock2006motivational}; dan (3) Berapa biayanya? Ini termasuk pertimbangan biaya langsung dari tertangkap tetapi juga beban psikologis dari perilaku akademik yang tidak jujur \citep{bertram2017academic} \citep{etgar2019white} \citep{murdock2006motivational}.

Ketidakjujuran akademik ditemukan berkorelasi positif dengan motivasi ekstrinsik \citep{barbaranelli2018machiavellian} \citep{grira2019rationality} \citep{krou2020achievement} \citep{murdock2006motivational} dan berkorelasi negatif dengan motivasi intrinsik \citep{foltynek2018analysis} \citep{barbaranelli2018machiavellian} \citep{grira2019rationality} \citep{murdock2006motivational}. Akan tetapi, bukan hanya motivasi yang mempengaruhi perilaku tersebut.

Literatur di bidang ini menunjukkan bahwa persepsi siswa tentang ketidakjujuran akademik dapat menjelaskan beberapa perilaku tersebut \citep{kiekkas2020reasons}. Pernyataan seperti "itu bukan masalah besar", "ini tidak benar-benar menyontek", "ini salah guru saya", atau "semua orang curang" \citep{stephens2007does} adalah contoh persepsi siswa tentang kurangnya integritas sebagai sesuatu yang tidak serius. Selain itu, siswa tidak selalu mempersepsikan perilaku tertentu, seperti penggunaan bahan tanpa memperhatikan sumbernya \citep{moss2018systematic} atau penggunaan catatan tersembunyi dalam ujian, sebagai karakteristik perilaku ketidakjujuran akademik \citep{kiekkas2020reasons}.

Berbeda dengan sikap siswa, staf pengajar menganggap ketidakjujuran akademik jauh lebih serius \citep{blau2021violation} \citep{stevens2013promoting}. \citet*{pincus2003faculty} bahkan menemukan bahwa dosen menganggap perilaku seperti menyalin dalam ujian, penggunaan materi terlarang selama ujian, mengikuti ujian untuk orang lain dan membayar seseorang untuk menulis makalah sebagai perilaku yang sangat tidak pantas. Kegagalan untuk berkontribusi pada kerja kelompok, berbohong dan menyajikan makalah yang sama di lebih dari satu mata kuliah adalah semua perilaku yang dianggap kurang parah oleh dosen. Namun, secara umum, dosen menganggap semua perilaku yang menunjukkan ketidakjujuran akademik lebih serius daripada mahasiswa, dan menganggap lebih banyak perilaku sebagai manifestasi ketidakjujuran akademik daripada mahasiswa.

Ketidakjujuran akademik telah menyibukkan akademisi selama bertahun-tahun, tetapi fenomena ini telah meningkat dalam beberapa tahun terakhir. Salah satu alasan peningkatan ini adalah pertumbuhan pengajaran online, dan teknologi yang memfasilitasi perilaku ini \citep{etgar2019white} \citep{peytcheva2018impact} \citep{sarwar2018paid}.

Dalam dekade terakhir, pendekatan pembelajaran yang inovatif telah diperkenalkan ke dalam sistem pendidikan tinggi. Perkembangan teknologi dan penggunaannya yang umum telah mendorong institusi pendidikan tinggi untuk memperkenalkan kursus online, baik kursus online atau hybrid, ke dalam program pembelajaran akademik mereka \citep{lee2017online} \citep{marshall2017attack}. Pendekatan ini memungkinkan peningkatan akses penuh dan mudah ke konten pembelajaran, penggunaan media sosial, Wikipedia, situs berbagi, dll \citep{ahmed2018student} \citep{lee2017online} \citep{peytcheva2018impact}. Bahkan, teknologi digital seperti Smartphone, komputer palm, komputer mobile dan PC dan Internet memungkinkan lebih banyak fleksibilitas, kreativitas dan kadang-kadang bahkan akurasi dan efektivitas. Oleh karena itu, mereka membantu proses belajar-mengajar, karena memungkinkan fotografi dan penyimpanan berbagai materi pembelajaran \citep{peytcheva2018impact} \citep{stephens2007does}, berbagi pengetahuan dan integrasi berbagai metode untuk membuat pembelajaran lebih aktif dan terlibat. 

Namun, integrasi kursus online tanpa mengintegrasikan aturan untuk perilaku etis yang sesuai untuk lingkungan online dan teknik khusus untuk mencegah ketidakjujuran akademik memberikan lahan subur bagi peningkatan frekuensi perilaku akademik yang tidak pantas \citep{marshall2017attack}.Selain itu, keuntungan mengintegrasikan teknologi dalam pembelajaran (kenyamanan, fleksibilitas dan akses ke informasi) menjadi insentif terbesar untuk perilaku tidak jujur \citep{blau2017ethical} \citep{muhammad2020factors} \citep{peytcheva2018impact}. Contohnya adalah plagiarisme, yang—karena akses informasi yang mudah—menjadi mudah digunakan dengan copy-paste, jauh lebih mudah daripada menyalin \citep{sidi2019ethical}.

Penelitian menunjukkan bahwa siswa mulai menggunakan alat teknologi yang tidak sah seperti Smartwatch dan Smartphone untuk perilaku ini \citep{birks2020managing} \citep{blau2017ethical}. Kemudian juga ditemukan bahwa dosen dan mahasiswa sama-sama percaya bahwa menyontek lebih mudah di kursus online \citep{kennedy2000academic}.

Dalam lingkungan belajar tanpa pengawasan, ada penjelasan tambahan untuk perilaku tidak etis setelah integrasi teknologi \citep{peytcheva2018impact}, seperti kelebihan aplikasi berbasis internet yang dapat diakses oleh mahasiswa, akses mudah ke dukungan tidak sah dari luar kampus (outsourcing), interaksi tatap muka yang tidak memadai dengan staf pengajar dalam kursus online yang mengarah pada penurunan komitmen moral, umpan balik yang tidak memadai tentang kegiatan belajar akademik, pedoman yang tidak tepat bagi siswa dalam perjalanan belajar online, kurangnya pelatihan yang sesuai untuk pembelajaran online dan kurangnya mekanisme pemantauan yang tepat \citep{von2001can}. Namun demikian, perlu dicatat bahwa teknologi itu sendiri bukanlah penyebab perilaku tidak jujur, itu hanya mempermudah dan memungkinkannya terjadi \citep{blau2017ethical} \citep{etgar2019white} \citep{sarwar2018paid}. Ada juga beberapa teknik yang dirancang khusus yang efektif untuk mencegah ketidakjujuran akademik, faktor yang juga dapat menjelaskan peningkatan prevalensi perilaku akademik yang tidak pantas \citep{marshall2017attack}.

Penelitian juga menunjukkan bahwa banyak perilaku yang dianggap sebagai ketidakjujuran akademik dan terkait dengan perangkat digital berasal dari kurangnya pengetahuan dan pemahaman siswa tentang perilaku etis \citep{blau2017ethical}. Misalnya, "copy-paste" tidak selalu dianggap sebagai praktik yang tidak etis. Poin penting lainnya adalah persepsi akademik tentang ketidakjujuran akademik di ruang digital kurang berbahaya daripada ketidakjujuran akademik di ruang akademik analog, karena di ruang digital dianggap sebagai "kejahatan kerah putih" \citep{etgar2019white} dan karena itu dianggap kurang merusak. Dengan demikian, hukuman untuk perilaku ini lebih ringan dibandingkan dengan hukuman di lembaga akademis yang serupa \citep{etgar2019white}.
  \section{Metode Mengurangi Ketidakjujuran Akademik dalam Penilaian Online}
\label{sec:antiketidakjujuranakademik}

Sama seperti alasan mengapa siswa menyontek bervariasi, demikian juga metode untuk mengurangi ketidakjujuran akademik. Kami kembali mengatur topik dalam kaitannya dengan faktor-faktor yang terkait dengan individu siswa, institusi, media penyampaian, dan penilaian itu sendiri. Sepanjang, kami fokus terutama pada penilaian sumatif yang mungkin memiliki berbagai format, dari pertanyaan pilihan ganda hingga esai buku terbuka yang dibawa pulang. Meskipun metode untuk mencegah kecurangan dibahas secara terpisah dari alasan mengapa siswa menyontek dalam makalah ini, kami menekankan bahwa metode tersebut harus dipertimbangkan bersama dengan pertimbangan alasan dan motivasi bahwa siswa mungkin terlibat dalam ketidakjujuran akademik.

\subsection{Metode level individual dan institusi}
\label{subsec:levelindividuinstitusi}

Dibahas metode tingkat individu dan universitas untuk mengurangi ketidakjujuran akademik bersama di sini, karena metode saat ini mempertimbangkan pengaruh dua arah dari setiap tingkat. Seperti yang disoroti dalam Faktor Kelembagaan, faktor kelembagaan yang dapat meningkatkan ketidakjujuran akademik termasuk sanksi yang longgar atau tidak memadai atas ketidakjujuran akademik, pengetahuan yang tidak memadai tentang kebijakan dan standar di seluruh siswa, instruktur, dan administrator, dan upaya yang tidak memadai untuk memberi tahu siswa tentang kebijakan dan standar ini \citep{akbulut2008exploring} \citep{jordan2001college}. Untuk memastikan kejujuran akademik di universitas, administrator dan staf harus secara jelas mendefinisikan ketidakjujuran akademik dan perilaku apa yang dianggap tidak jujur secara akademis. Siswa sering menunjukkan kebingungan tentang apa yang merupakan ketidakjujuran akademik, dan tanpa definisi yang jelas, banyak siswa mungkin menyontek tanpa mempertimbangkan perilaku mereka sebagai ketidakjujuran akademis. Dengan demikian, semakin banyak anggota fakultas mendiskusikan kejujuran akademik, semakin sedikit ambiguitas yang dimiliki siswa ketika menghadapi contoh ketidakjujuran akademik \citep{tatum2017honor}. Selain menyadarkan siswa tentang apa yang dimaksud dengan ketidakjujuran akademik, penting juga untuk menyadarkan siswa tentang hukuman yang ada untuk perilaku tidak jujur secara akademis. Ketidakjujuran akademik berbanding terbalik dengan tingkat keparahan yang dirasakan dari hukuman universitas untuk perilaku tidak jujur secara akademis \citep{mccabe2002honor}. Ketika anggota fakultas menyadari kebijakan institusi mereka terhadap ketidakjujuran akademik dan mengatasi semua contoh ketidakjujuran, lebih sedikit perilaku tidak jujur yang terjadi secara akademis \citep{boehm2009promoting}.

\subsection{Metode Terkait dengan Media Penyampaian}
\label{subsec:metodemediapenyampaian}

Berbagai metode untuk memerangi ketidakjujuran akademik dalam penilaian online berfokus pada cara penilaian disampaikan dan diawasi. Satu pandangan adalah bahwa ujian sumatif yang diawasi secara langsung di pusat pengujian adalah praktik terbaik untuk kursus online karena potensi kemudahan menyontek di lingkungan yang tidak diawasi atau lingkungan yang diawasi secara online \citep{edling2000information} \citep{rovai2000online} \citep{deal2002distance}.

\paragraph{Deteksi kecurangan online}
\label{par:deteksikecuranganonline}

Sistem deteksi kecurangan ujian yang dijelaskan di bawah ini telah dikembangkan, sebagian, karena mengadakan ujian secara langsung di lokasi terdaftar dengan pengawas langsung seringkali tidak memungkinkan karena alasan keuangan, perjalanan, atau logistik lainnya \citep{cluskey2011thwarting}. 

Meskipun pengawasan online memberikan beberapa keuntungan intuitif untuk mendeteksi perilaku curang, dan ini memetakan secara dekat ke proses pengawasan tatap muka yang sudah dikenal, banyak yang telah mengangkat kekhawatiran di media tentang etika dan kemanjuran sistem ini. Misalnya, kekhawatiran telah dikemukakan tentang pelanggaran privasi dan perlindungan data oleh siswa \citep{lawson2020schools}, dan pelanggaran telah terjadi (contohnya \citet{holden2021academic}). Tidak hanya ada kekhawatiran tentang etika mengenai perangkat lunak pengawasan online, tetapi ada juga kekhawatiran tentang apakah metode ini efektif, dan jika demikian, untuk berapa lama. Misalnya, sudah lama tersedia panduan yang menunjukkan cara "menipu" perangkat lunak curang (contohnya \citet{binstein2015knuckle}). Jika seorang instruktur menganggap pengawasan online efektif dan perlu, sebelum menggunakan pengawasan online, instruktur harus secara eksplisit mempertimbangkan apakah siswa diperlakukan secara adil dan setara, seperti yang seharusnya mereka lakukan dalam interaksi apa pun dengan siswa. Instruktur juga didorong untuk menyelidiki dengan hati-hati kebijakan privasi yang terkait dengan perangkat lunak pendeteksi kecurangan online, dan kebijakan institusi apa pun yang berlaku (misalnya, kebijakan akses dan penyimpanan data), sebelum menggunakan teknologi tersebut. 


\paragraph{Peringkasan video}
\label{par:peringkasanvideo}
Perangkat lunak peringkasan video, juga disebut sebagai abstraksi video, menggunakan kecerdasan buatan untuk mendeteksi kejadian menyontek yang mungkin terjadi selama ujian \citep{truong2007video}. Siswa direkam video menggunakan webcam mereka sendiri selama ujian. Jika acara curang terdeteksi, program akan menandai video untuk dilihat di masa mendatang oleh pengawas. Dengan demikian, tuntutan waktu pengawas berkurang, namun siswa dipantau. Program peringkasan video dapat menghasilkan bingkai utama (kumpulan gambar yang diekstrak dari sumber video) atau skim video (segmen video yang diekstrak dari sumber video) untuk menunjukkan potensi perilaku curang (contohnya \citet{truong2007video}). Kedua bentuk ini menyampaikan potensi kecurangan dalam rangka penentuan masa depan oleh pengawas manusia. Namun, video skim memiliki keunggulan dibandingkan keyframe karena memiliki kemampuan untuk memasukkan elemen audio dan gerakan yang menyampaikan informasi terkait dalam proses pengawasan \citep{cote2016video}. Keuntungan utama memilih layanan pengawasan seperti ini adalah mengurangi jam yang harus dilakukan pengawas untuk mengawasi ujian. Namun, mendeteksi perilaku menyontek tanpa interaksi manusia secara langsung adalah proses yang sulit. Memodelkan perilaku mencurigakan itu kompleks karena perilaku menyontek biasanya tidak mengikuti pola atau jenis, sehingga sulit untuk dikenali secara akurat \citep{cote2016video}.

\paragraph{Rekaman Web video}
\label{par:rekamanwebvideo}

Sehubungan dengan ujian online, perekaman video web mengacu pada situasi di mana siswa merekam video selama keseluruhan ujian untuk dilihat nanti oleh instruktur. Seperti metode ringkasan video, perangkat lunak pendeteksi dapat digunakan untuk menandai aktivitas mencurigakan apa pun untuk dilihat nanti. Administrator dan instruktur mungkin merasa lebih percaya diri dalam layanan ini karena mereka dapat melihat seluruh ujian, tidak hanya contoh yang ditandai. Namun, meninjau semua ujian secara individual mungkin tidak layak, dan sebagian besar ujian tidak ditinjau secara penuh. Tidak seperti program ringkasan video, program perekaman video web tidak memiliki pengawas khusus yang meninjau semua contoh yang ditandai, dan sebaliknya mengandalkan tinjauan oleh administrator dan instruktur itu sendiri. Mengetahui bahwa rekaman sedang terjadi dapat menghalangi siswa, tetapi seperti halnya deteksi berdasarkan kecerdasan buatan, tidak menjamin bahwa semua perilaku menyontek akan terdeteksi. Penting untuk dicatat bahwa dengan metode ini, seperti metode sebelumnya, tidak ada peluang untuk intervensi oleh pengawas jika suatu peristiwa ditandai sebagai kemungkinan pelanggaran integritas akademik. Dengan demikian, mungkin ada situasi ambigu yang telah ditandai secara elektronik tanpa kesempatan untuk menyelidiki lebih lanjut, dan kehilangan peluang untuk pencegahan.

\paragraph{Pengawasan Online Langsung}
\label{par:pengawasanonlinelangsung}

Jenis terakhir dari pengawasan online, dan bisa dibilang yang paling ketat, disebut sebagai pengawasan online langsung atau pengawasan konferensi video web. Metode ini menggunakan webcam dan mikrofon siswa untuk memungkinkan pengawas langsung mengawasi siswa selama ujian online. Layanan dapat berkisar dari sesi pengawasan satu lawan satu hingga sesi pengawasan kelompok di mana satu pengawas mengawasi banyak siswa. Banyak administrator mungkin merasa paling nyaman menggunakan layanan semacam ini karena paling dekat dengan ujian yang diawasi secara langsung. Namun, bahkan dengan pengawas langsung yang mengawasi siswa, perilaku menyontek bisa tidak terdeteksi. Pada awal sesi, siswa biasanya diminta untuk menunjukkan lingkungan pengujian mereka kepada pengawas mereka; Namun, materi menyontek dapat ditarik keluar saat ujian tanpa diketahui di lingkungan sekitar. Jika pengawas tidak mencurigai perilaku curang, mereka tidak akan meminta pandangan lain dari seluruh ruangan. Pengawasan online langsung juga biasanya merupakan opsi yang paling mahal.

\paragraph{Biometrik}
\label{par:biometrik}

Penggunaan biometrik, pengukuran fitur fisiologis atau perilaku individu, adalah metode otentikasi yang memungkinkan verifikasi identitas berkelanjutan \citep{rabuzin2006learning} \citep{cote2016video}. Metode otentikasi ini membandingkan sampel biometrik terdaftar dengan biometrik yang baru diambil untuk mengidentifikasi siswa \citep{podio2001biometric}. Saat mempertimbangkan penggunaan data biometrik, potensi bias dalam identifikasi, keamanan data, dan privasi harus dipertimbangkan dengan cermat. Mungkin risiko yang terkait dengan penggunaan data biometrik, mengingat sifat dasar data ini, lebih besar daripada manfaat untuk penilaian.

Ada dua jenis utama fitur biometrik: fitur yang memerlukan kontak fisik langsung dengan pemindai, seperti sidik jari, dan fitur yang tidak memerlukan kontak fisik dengan pemindai seperti warna rambut \citep{rabuzin2006learning}. Biometrik umumnya menggunakan ciri-ciri "lunak" seperti tinggi, berat, usia, jenis kelamin, dan etnis, karakteristik fisiologis seperti mata, dan wajah, dan karakteristik perilaku seperti dinamika penekanan tombol, gerakan mouse, dan tanda tangan \citep{cerimagic2019online} \citep{rabuzin2006learning}.

Identifikasi berbasis biometrik sering lebih disukai daripada metode lain karena fitur biometrik tidak dapat dipalsukan, dilupakan, atau hilang, tidak seperti kata sandi dan kartu identitas \citep{prabhakar2002decision}\citep{rudrapal2012voice}. Namun, fitur biometrik yang dipertimbangkan harus universal, unik, permanen, terukur, akurat, dan dapat diterima \citep{frischholz2000biold}. Secara khusus, fitur biometrik yang ideal harus permanen dan tidak dapat diubah, dan prosedur pengumpulan fitur harus tidak mencolok dan dilakukan oleh perangkat yang memerlukan sedikit atau tanpa kontak. Selanjutnya, sistem idealnya otomatis, sangat akurat, dan beroperasi secara real time. Sistem biometrik multimodal menggunakan beberapa ciri dan teknologi biometrik secara bersamaan untuk memverifikasi identitas pengguna. Sistem multimodal cenderung lebih akurat, karena menggabungkan dua atau lebih fitur meningkatkan akurasi pengenalan.

Pengenalan sidik jari adalah salah satu fitur biometrik yang paling banyak digunakan karena merupakan pengidentifikasi unik \citep{aggarwal2008gradient} dan memiliki sejarah penggunaan di berbagai bidang profesional, terutama oleh polisi. Selain itu, sidik jari telah menjadi pengenal yang umum digunakan untuk perangkat genggam pribadi seperti telepon. Namun, penggunaan biometrik sidik jari untuk identifikasi siswa selama ujian online dapat memerlukan sumber daya tambahan seperti pemindai sidik jari, ponsel yang dilengkapi dengan teknologi sidik jari, atau perangkat lunak lain di lokasi siswa, yang dapat membatasi kepraktisannya saat ini \citep{ullah2012using}. Demikian pula, pengenalan wajah menggunakan pengenalan gambar dan algoritma pencocokan pola untuk mengotentikasi identitas siswa. Biometrik ini juga merupakan kandidat yang baik untuk ujian online; namun, itu mungkin tidak selalu dapat diandalkan karena kompleksitas teknologi pengenalan dan variabilitas dalam pencahayaan, rambut wajah, dan fitur wajah.

\paragraph{Pengecekan originalitas text}
\label{par:originalitastext}

Saat menggunakan penilaian yang memerlukan jawaban tertulis, perangkat lunak yang memeriksa keaslian teks (seperti "TurnItIn") dapat membantu mengidentifikasi karya yang diambil dari sumber tanpa kutipan yang tepat. Dengan metode ini, karya yang dikirimkan dibandingkan dengan karya lain yang disimpan di bank perangkat lunak untuk memeriksa keasliannya. Manfaat metode ini termasuk dapat membandingkan pekerjaan yang dikirimkan dengan pekerjaan yang tersedia untuk umum (seperti yang ditentukan oleh perusahaan perangkat lunak) untuk memeriksa tingkat tumpang tindih yang penting, serta membandingkan pekerjaan yang dikirimkan dengan tugas lain yang telah dikirimkan sebelumnya.

Meskipun memeriksa orisinalitas teks dapat membantu dalam mendeteksi plagiarisme yang tidak disengaja dan disengaja, ada kekhawatiran tentang etika praktik ini, termasuk pelanggaran hak cipta karya siswa \citep{horovitz2008two}. Instruktur biasanya dapat menentukan dalam perangkat lunak apakah pekerjaan yang dikirimkan akan disimpan untuk perbandingan nanti (atau tidak), dan informasi ini, bersama dengan kebijakan penggunaan yang lebih luas, harus dimasukkan secara khusus dalam silabus atau komunikasi relevan lainnya dengan siswa. Selain itu, saat menggunakan perangkat lunak pemeriksa orisinalitas, penting untuk diketahui bahwa tumpang tindih yang tinggi dengan karya lain tidak selalu menunjukkan karya yang dijiplak, dan mungkin ada tingkat positif palsu yang tinggi. Misalnya, pengiriman dengan tingkat referensi yang sesuai tinggi dapat menghasilkan skor tinggi untuk tumpang tindih hanya karena referensi tersebut standar di banyak karya. Dengan demikian, instruktur harus mengacu pada laporan orisinalitas lengkap sehingga mereka dapat menggunakan penilaian apakah skor tinggi benar-benar mencerminkan plagiarisme.

\paragraph{Penguncian browser}
\label{par:penguncianbrowser}

Penguncian browser mencegah penggunaan materi elektronik tambahan selama ujian dengan memblokir siswa dari mengunjungi situs web eksternal atau menggunakan aplikasi yang tidak sah pada perangkat yang sama dengan yang digunakan untuk mengikuti penilaian \citep{cote2016video}. Program-program ini mengendalikan seluruh sistem komputer dengan melarang akses ke pengelola tugas, fungsi salin dan tempel, dan tombol fungsi pada perangkat itu \citep{percival2008virtual}. Meskipun mungkin membantu, browser penguncian tidak dapat menjamin bahwa informasi eksternal tidak akan diakses. Siswa masih dapat mengakses informasi menggunakan komputer lain, ponsel, catatan kelas, dll, selama penilaian. Selain menggunakan materi eksternal, siswa juga dapat menyontek dengan membuat program browser penguncian tidak beroperasi \citep{percival2008virtual}. Untuk alasan ini, diusulkan agar program ini digunakan bersama dengan langkah-langkah keamanan ujian lainnya untuk mencegah dan mendeteksi perilaku menyontek selama ujian.

\paragraph{Pertanyaan tantangan}
\label{par:pertanyaantantangan}

Tantangan atau pertanyaan keamanan adalah salah satu metode paling sederhana untuk mengautentikasi peserta tes. Metode ini membutuhkan pengetahuan pribadi untuk mengotentikasi siswa dan disebut sebagai metode 'otentikasi berbasis pengetahuan'. Siswa diberi pertanyaan pilihan ganda berdasarkan sejarah pribadi mereka, seperti informasi tentang alamat rumah masa lalu mereka, nama sekolah menengah mereka, atau nama gadis ibu mereka. Siswa harus menjawab pertanyaan-pertanyaan ini untuk mengakses ujian, dan pertanyaan juga dapat ditanyakan secara acak selama penilaian . Pertanyaan-pertanyaan ini sering kali didasarkan pada data pihak ketiga menggunakan sistem data mining atau dapat dimasukkan oleh seorang siswa pada log-in awal sebelum ujian apa pun. Ketika seorang siswa meminta ujian, pertanyaan tantangan dihasilkan secara acak dari pertanyaan pengaturan profil awal atau informasi pihak ketiga, dan jawaban dibandingkan untuk memverifikasi identitas siswa. Metode yang relatif sederhana ini dapat digunakan untuk mengautentikasi peserta tes; namun, itu tidak dapat digunakan untuk memantau perilaku siswa selama ujian. Selain itu, siswa mungkin masih dapat melewati proses otentikasi dengan memberikan jawaban kepada orang lain agar orang lain mengikuti ujian, atau berkolaborasi dengan orang lain saat mengerjakan ujian. Jadi, jika dipilih, metode ini harus digunakan bersama dengan metode keamanan ujian lainnya untuk memastikan kejujuran akademis.

  % Ubah judul dan label berikut sesuai dengan yang diinginkan.
\section{Kesimpulan}
\label{sec:kesimpulan}

% Ubah paragraf-paragraf pada bagian ini sesuai dengan yang diinginkan.

Makalah ini dimulai dengan memberikan tinjauan pemikiran terkini mengenai alasan mengapa siswa mungkin merasa termotivasi untuk terlibat dalam perilaku yang melanggar integritas akademik. Kami mendekati pertanyaan ini dengan mempertimbangkan empat "tingkat" untuk mempertimbangkan integritas akademik: siswa, institusi, media penyampaian, dan penilaian. Kami menyarankan bahwa ketika memeriksa integritas akademik di lingkungan online, akan diperlukan penelitian lanjutan yang mengeksplorasi budaya menyontek dan sifat, dan motivasi untuk, menyontek pada berbagai jenis penilaian. Lebih lanjut, seperti yang ditunjukkan, penelitian hingga saat ini telah menghasilkan temuan yang beragam dalam kaitannya dengan apakah ketidakjujuran akademik mungkin lebih atau kurang lazim di lingkungan online, dan kami telah menyerukan penelitian lebih lanjut yang meneliti jenis penilaian, bidang studi, dan demografi siswa (mis. , usia dan alasan mendaftar di kursus). Di paruh kedua ulasan ini, kami merinci metode untuk mencegah dan mendeteksi perilaku menyontek, dengan fokus pada penilaian sumatif online. Kami menekankan lagi, bahwa metode ini harus dipertimbangkan bersama dengan pertimbangan yang lebih luas dari alasan dan motivasi bahwa siswa mungkin terlibat dalam ketidakjujuran akademik di tempat pertama, dan dengan perhatian eksplisit dan perhatian terhadap privasi siswa dan perlakuan yang adil.

Tantangan atau pertanyaan keamanan adalah salah satu metode paling sederhana untuk mengautentikasi peserta tes. Metode ini membutuhkan pengetahuan pribadi untuk mengotentikasi siswa dan disebut sebagai metode 'otentikasi berbasis pengetahuan'. Siswa diberi pertanyaan pilihan ganda berdasarkan sejarah pribadi mereka, seperti informasi tentang alamat rumah masa lalu mereka, nama sekolah menengah mereka, atau nama gadis ibu mereka. Siswa harus menjawab pertanyaan-pertanyaan ini untuk mengakses ujian, dan pertanyaan juga dapat ditanyakan secara acak selama penilaian .

Integritas akademik tetap menjadi elemen integral dari pendidikan tinggi. Nilai-nilai prinsip yang membentuk integritas akademik tidak hanya menjunjung tinggi reputasi universitas dan nilai serta makna gelar yang dianugerahkannya, tetapi juga menciptakan kerangka kerja bersama untuk pekerjaan profesional yang diperluas di luar akademi. Oleh karena itu, karena studi online terus berkembang dalam pendidikan pasca sekolah menengah, kami percaya bahwa penting untuk mengembangkan beasiswa dan diskusi mengenai pemeliharaan integritas akademik di lingkungan online.


  % Menyeimbangkan bagian akhir di kedua kolom
  \balance

  % Menampilkan daftar pustaka dengan format IEEE
  \bibliographystyle{IEEEtranN}
  \bibliography{pustaka/pustaka.bib}

\end{document}
