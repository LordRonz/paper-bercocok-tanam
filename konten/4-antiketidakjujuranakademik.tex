\section{Metode Mengurangi Ketidakjujuran Akademik dalam Penilaian Online}
\label{sec:antiketidakjujuranakademik}

Sama seperti alasan mengapa siswa menyontek bervariasi, demikian juga metode untuk mengurangi ketidakjujuran akademik. Kami kembali mengatur topik dalam kaitannya dengan faktor-faktor yang terkait dengan individu siswa, institusi, media penyampaian, dan penilaian itu sendiri. Sepanjang, kami fokus terutama pada penilaian sumatif yang mungkin memiliki berbagai format, dari pertanyaan pilihan ganda hingga esai buku terbuka yang dibawa pulang. Meskipun metode untuk mencegah kecurangan dibahas secara terpisah dari alasan mengapa siswa menyontek dalam makalah ini, kami menekankan bahwa metode tersebut harus dipertimbangkan bersama dengan pertimbangan alasan dan motivasi bahwa siswa mungkin terlibat dalam ketidakjujuran akademik.

\subsection{Metode level individual dan institusi}
\label{subsec:levelindividuinstitusi}

Dibahas metode tingkat individu dan universitas untuk mengurangi ketidakjujuran akademik bersama di sini, karena metode saat ini mempertimbangkan pengaruh dua arah dari setiap tingkat. Seperti yang disoroti dalam Faktor Kelembagaan, faktor kelembagaan yang dapat meningkatkan ketidakjujuran akademik termasuk sanksi yang longgar atau tidak memadai atas ketidakjujuran akademik, pengetahuan yang tidak memadai tentang kebijakan dan standar di seluruh siswa, instruktur, dan administrator, dan upaya yang tidak memadai untuk memberi tahu siswa tentang kebijakan dan standar ini \citep{akbulut2008exploring} \citep{jordan2001college}. Untuk memastikan kejujuran akademik di universitas, administrator dan staf harus secara jelas mendefinisikan ketidakjujuran akademik dan perilaku apa yang dianggap tidak jujur secara akademis. Siswa sering menunjukkan kebingungan tentang apa yang merupakan ketidakjujuran akademik, dan tanpa definisi yang jelas, banyak siswa mungkin menyontek tanpa mempertimbangkan perilaku mereka sebagai ketidakjujuran akademis. Dengan demikian, semakin banyak anggota fakultas mendiskusikan kejujuran akademik, semakin sedikit ambiguitas yang dimiliki siswa ketika menghadapi contoh ketidakjujuran akademik \citep{tatum2017honor}. Selain menyadarkan siswa tentang apa yang dimaksud dengan ketidakjujuran akademik, penting juga untuk menyadarkan siswa tentang hukuman yang ada untuk perilaku tidak jujur secara akademis. Ketidakjujuran akademik berbanding terbalik dengan tingkat keparahan yang dirasakan dari hukuman universitas untuk perilaku tidak jujur secara akademis \citep{mccabe2002honor}. Ketika anggota fakultas menyadari kebijakan institusi mereka terhadap ketidakjujuran akademik dan mengatasi semua contoh ketidakjujuran, lebih sedikit perilaku tidak jujur yang terjadi secara akademis \citep{boehm2009promoting}.

\subsection{Metode Terkait dengan Media Penyampaian}
\label{subsec:metodemediapenyampaian}

Berbagai metode untuk memerangi ketidakjujuran akademik dalam penilaian online berfokus pada cara penilaian disampaikan dan diawasi. Satu pandangan adalah bahwa ujian sumatif yang diawasi secara langsung di pusat pengujian adalah praktik terbaik untuk kursus online karena potensi kemudahan menyontek di lingkungan yang tidak diawasi atau lingkungan yang diawasi secara online \citep{edling2000information} \citep{rovai2000online} \citep{deal2002distance}.

\paragraph{Deteksi kecurangan online}
\label{par:deteksikecuranganonline}

Sistem deteksi kecurangan ujian yang dijelaskan di bawah ini telah dikembangkan, sebagian, karena mengadakan ujian secara langsung di lokasi terdaftar dengan pengawas langsung seringkali tidak memungkinkan karena alasan keuangan, perjalanan, atau logistik lainnya \citep{cluskey2011thwarting}. 

Meskipun pengawasan online memberikan beberapa keuntungan intuitif untuk mendeteksi perilaku curang, dan ini memetakan secara dekat ke proses pengawasan tatap muka yang sudah dikenal, banyak yang telah mengangkat kekhawatiran di media tentang etika dan kemanjuran sistem ini. Misalnya, kekhawatiran telah dikemukakan tentang pelanggaran privasi dan perlindungan data oleh siswa \citep{lawson2020schools}, dan pelanggaran telah terjadi (contohnya \citet{holden2021academic}). Tidak hanya ada kekhawatiran tentang etika mengenai perangkat lunak pengawasan online, tetapi ada juga kekhawatiran tentang apakah metode ini efektif, dan jika demikian, untuk berapa lama. Misalnya, sudah lama tersedia panduan yang menunjukkan cara "menipu" perangkat lunak curang (contohnya \citet{binstein2015knuckle}). Jika seorang instruktur menganggap pengawasan online efektif dan perlu, sebelum menggunakan pengawasan online, instruktur harus secara eksplisit mempertimbangkan apakah siswa diperlakukan secara adil dan setara, seperti yang seharusnya mereka lakukan dalam interaksi apa pun dengan siswa. Instruktur juga didorong untuk menyelidiki dengan hati-hati kebijakan privasi yang terkait dengan perangkat lunak pendeteksi kecurangan online, dan kebijakan institusi apa pun yang berlaku (misalnya, kebijakan akses dan penyimpanan data), sebelum menggunakan teknologi tersebut. 


\paragraph{Peringkasan video}
\label{par:peringkasanvideo}
Perangkat lunak peringkasan video, juga disebut sebagai abstraksi video, menggunakan kecerdasan buatan untuk mendeteksi kejadian menyontek yang mungkin terjadi selama ujian \citep{truong2007video}. Siswa direkam video menggunakan webcam mereka sendiri selama ujian. Jika acara curang terdeteksi, program akan menandai video untuk dilihat di masa mendatang oleh pengawas. Dengan demikian, tuntutan waktu pengawas berkurang, namun siswa dipantau. Program peringkasan video dapat menghasilkan bingkai utama (kumpulan gambar yang diekstrak dari sumber video) atau skim video (segmen video yang diekstrak dari sumber video) untuk menunjukkan potensi perilaku curang (contohnya \citet{truong2007video}). Kedua bentuk ini menyampaikan potensi kecurangan dalam rangka penentuan masa depan oleh pengawas manusia. Namun, video skim memiliki keunggulan dibandingkan keyframe karena memiliki kemampuan untuk memasukkan elemen audio dan gerakan yang menyampaikan informasi terkait dalam proses pengawasan \citep{cote2016video}. Keuntungan utama memilih layanan pengawasan seperti ini adalah mengurangi jam yang harus dilakukan pengawas untuk mengawasi ujian. Namun, mendeteksi perilaku menyontek tanpa interaksi manusia secara langsung adalah proses yang sulit. Memodelkan perilaku mencurigakan itu kompleks karena perilaku menyontek biasanya tidak mengikuti pola atau jenis, sehingga sulit untuk dikenali secara akurat \citep{cote2016video}.

\paragraph{Rekaman Web video}
\label{par:rekamanwebvideo}

Sehubungan dengan ujian online, perekaman video web mengacu pada situasi di mana siswa merekam video selama keseluruhan ujian untuk dilihat nanti oleh instruktur. Seperti metode ringkasan video, perangkat lunak pendeteksi dapat digunakan untuk menandai aktivitas mencurigakan apa pun untuk dilihat nanti. Administrator dan instruktur mungkin merasa lebih percaya diri dalam layanan ini karena mereka dapat melihat seluruh ujian, tidak hanya contoh yang ditandai. Namun, meninjau semua ujian secara individual mungkin tidak layak, dan sebagian besar ujian tidak ditinjau secara penuh. Tidak seperti program ringkasan video, program perekaman video web tidak memiliki pengawas khusus yang meninjau semua contoh yang ditandai, dan sebaliknya mengandalkan tinjauan oleh administrator dan instruktur itu sendiri. Mengetahui bahwa rekaman sedang terjadi dapat menghalangi siswa, tetapi seperti halnya deteksi berdasarkan kecerdasan buatan, tidak menjamin bahwa semua perilaku menyontek akan terdeteksi. Penting untuk dicatat bahwa dengan metode ini, seperti metode sebelumnya, tidak ada peluang untuk intervensi oleh pengawas jika suatu peristiwa ditandai sebagai kemungkinan pelanggaran integritas akademik. Dengan demikian, mungkin ada situasi ambigu yang telah ditandai secara elektronik tanpa kesempatan untuk menyelidiki lebih lanjut, dan kehilangan peluang untuk pencegahan.

\paragraph{Pengawasan Online Langsung}
\label{par:pengawasanonlinelangsung}

Jenis terakhir dari pengawasan online, dan bisa dibilang yang paling ketat, disebut sebagai pengawasan online langsung atau pengawasan konferensi video web. Metode ini menggunakan webcam dan mikrofon siswa untuk memungkinkan pengawas langsung mengawasi siswa selama ujian online. Layanan dapat berkisar dari sesi pengawasan satu lawan satu hingga sesi pengawasan kelompok di mana satu pengawas mengawasi banyak siswa. Banyak administrator mungkin merasa paling nyaman menggunakan layanan semacam ini karena paling dekat dengan ujian yang diawasi secara langsung. Namun, bahkan dengan pengawas langsung yang mengawasi siswa, perilaku menyontek bisa tidak terdeteksi. Pada awal sesi, siswa biasanya diminta untuk menunjukkan lingkungan pengujian mereka kepada pengawas mereka; Namun, materi menyontek dapat ditarik keluar saat ujian tanpa diketahui di lingkungan sekitar. Jika pengawas tidak mencurigai perilaku curang, mereka tidak akan meminta pandangan lain dari seluruh ruangan. Pengawasan online langsung juga biasanya merupakan opsi yang paling mahal.

\paragraph{Biometrik}
\label{par:biometrik}

Penggunaan biometrik, pengukuran fitur fisiologis atau perilaku individu, adalah metode otentikasi yang memungkinkan verifikasi identitas berkelanjutan \citep{rabuzin2006learning} \citep{cote2016video}. Metode otentikasi ini membandingkan sampel biometrik terdaftar dengan biometrik yang baru diambil untuk mengidentifikasi siswa \citep{podio2001biometric}. Saat mempertimbangkan penggunaan data biometrik, potensi bias dalam identifikasi, keamanan data, dan privasi harus dipertimbangkan dengan cermat. Mungkin risiko yang terkait dengan penggunaan data biometrik, mengingat sifat dasar data ini, lebih besar daripada manfaat untuk penilaian.

Ada dua jenis utama fitur biometrik: fitur yang memerlukan kontak fisik langsung dengan pemindai, seperti sidik jari, dan fitur yang tidak memerlukan kontak fisik dengan pemindai seperti warna rambut \citep{rabuzin2006learning}. Biometrik umumnya menggunakan ciri-ciri "lunak" seperti tinggi, berat, usia, jenis kelamin, dan etnis, karakteristik fisiologis seperti mata, dan wajah, dan karakteristik perilaku seperti dinamika penekanan tombol, gerakan mouse, dan tanda tangan \citep{cerimagic2019online} \citep{rabuzin2006learning}.

Identifikasi berbasis biometrik sering lebih disukai daripada metode lain karena fitur biometrik tidak dapat dipalsukan, dilupakan, atau hilang, tidak seperti kata sandi dan kartu identitas \citep{prabhakar2002decision}\citep{rudrapal2012voice}. Namun, fitur biometrik yang dipertimbangkan harus universal, unik, permanen, terukur, akurat, dan dapat diterima \citep{frischholz2000biold}. Secara khusus, fitur biometrik yang ideal harus permanen dan tidak dapat diubah, dan prosedur pengumpulan fitur harus tidak mencolok dan dilakukan oleh perangkat yang memerlukan sedikit atau tanpa kontak. Selanjutnya, sistem idealnya otomatis, sangat akurat, dan beroperasi secara real time. Sistem biometrik multimodal menggunakan beberapa ciri dan teknologi biometrik secara bersamaan untuk memverifikasi identitas pengguna. Sistem multimodal cenderung lebih akurat, karena menggabungkan dua atau lebih fitur meningkatkan akurasi pengenalan.

Pengenalan sidik jari adalah salah satu fitur biometrik yang paling banyak digunakan karena merupakan pengidentifikasi unik \citep{aggarwal2008gradient} dan memiliki sejarah penggunaan di berbagai bidang profesional, terutama oleh polisi. Selain itu, sidik jari telah menjadi pengenal yang umum digunakan untuk perangkat genggam pribadi seperti telepon. Namun, penggunaan biometrik sidik jari untuk identifikasi siswa selama ujian online dapat memerlukan sumber daya tambahan seperti pemindai sidik jari, ponsel yang dilengkapi dengan teknologi sidik jari, atau perangkat lunak lain di lokasi siswa, yang dapat membatasi kepraktisannya saat ini \citep{ullah2012using}. Demikian pula, pengenalan wajah menggunakan pengenalan gambar dan algoritma pencocokan pola untuk mengotentikasi identitas siswa. Biometrik ini juga merupakan kandidat yang baik untuk ujian online; namun, itu mungkin tidak selalu dapat diandalkan karena kompleksitas teknologi pengenalan dan variabilitas dalam pencahayaan, rambut wajah, dan fitur wajah.

\paragraph{Pengecekan originalitas text}
\label{par:originalitastext}

Saat menggunakan penilaian yang memerlukan jawaban tertulis, perangkat lunak yang memeriksa keaslian teks (seperti "TurnItIn") dapat membantu mengidentifikasi karya yang diambil dari sumber tanpa kutipan yang tepat. Dengan metode ini, karya yang dikirimkan dibandingkan dengan karya lain yang disimpan di bank perangkat lunak untuk memeriksa keasliannya. Manfaat metode ini termasuk dapat membandingkan pekerjaan yang dikirimkan dengan pekerjaan yang tersedia untuk umum (seperti yang ditentukan oleh perusahaan perangkat lunak) untuk memeriksa tingkat tumpang tindih yang penting, serta membandingkan pekerjaan yang dikirimkan dengan tugas lain yang telah dikirimkan sebelumnya.

Meskipun memeriksa orisinalitas teks dapat membantu dalam mendeteksi plagiarisme yang tidak disengaja dan disengaja, ada kekhawatiran tentang etika praktik ini, termasuk pelanggaran hak cipta karya siswa \citep{horovitz2008two}. Instruktur biasanya dapat menentukan dalam perangkat lunak apakah pekerjaan yang dikirimkan akan disimpan untuk perbandingan nanti (atau tidak), dan informasi ini, bersama dengan kebijakan penggunaan yang lebih luas, harus dimasukkan secara khusus dalam silabus atau komunikasi relevan lainnya dengan siswa. Selain itu, saat menggunakan perangkat lunak pemeriksa orisinalitas, penting untuk diketahui bahwa tumpang tindih yang tinggi dengan karya lain tidak selalu menunjukkan karya yang dijiplak, dan mungkin ada tingkat positif palsu yang tinggi. Misalnya, pengiriman dengan tingkat referensi yang sesuai tinggi dapat menghasilkan skor tinggi untuk tumpang tindih hanya karena referensi tersebut standar di banyak karya. Dengan demikian, instruktur harus mengacu pada laporan orisinalitas lengkap sehingga mereka dapat menggunakan penilaian apakah skor tinggi benar-benar mencerminkan plagiarisme.

\paragraph{Penguncian browser}
\label{par:penguncianbrowser}

Penguncian browser mencegah penggunaan materi elektronik tambahan selama ujian dengan memblokir siswa dari mengunjungi situs web eksternal atau menggunakan aplikasi yang tidak sah pada perangkat yang sama dengan yang digunakan untuk mengikuti penilaian \citep{cote2016video}. Program-program ini mengendalikan seluruh sistem komputer dengan melarang akses ke pengelola tugas, fungsi salin dan tempel, dan tombol fungsi pada perangkat itu \citep{percival2008virtual}. Meskipun mungkin membantu, browser penguncian tidak dapat menjamin bahwa informasi eksternal tidak akan diakses. Siswa masih dapat mengakses informasi menggunakan komputer lain, ponsel, catatan kelas, dll, selama penilaian. Selain menggunakan materi eksternal, siswa juga dapat menyontek dengan membuat program browser penguncian tidak beroperasi \citep{percival2008virtual}. Untuk alasan ini, diusulkan agar program ini digunakan bersama dengan langkah-langkah keamanan ujian lainnya untuk mencegah dan mendeteksi perilaku menyontek selama ujian.

\paragraph{Pertanyaan tantangan}
\label{par:pertanyaantantangan}

Tantangan atau pertanyaan keamanan adalah salah satu metode paling sederhana untuk mengautentikasi peserta tes. Metode ini membutuhkan pengetahuan pribadi untuk mengotentikasi siswa dan disebut sebagai metode 'otentikasi berbasis pengetahuan'. Siswa diberi pertanyaan pilihan ganda berdasarkan sejarah pribadi mereka, seperti informasi tentang alamat rumah masa lalu mereka, nama sekolah menengah mereka, atau nama gadis ibu mereka. Siswa harus menjawab pertanyaan-pertanyaan ini untuk mengakses ujian, dan pertanyaan juga dapat ditanyakan secara acak selama penilaian . Pertanyaan-pertanyaan ini sering kali didasarkan pada data pihak ketiga menggunakan sistem data mining atau dapat dimasukkan oleh seorang siswa pada log-in awal sebelum ujian apa pun. Ketika seorang siswa meminta ujian, pertanyaan tantangan dihasilkan secara acak dari pertanyaan pengaturan profil awal atau informasi pihak ketiga, dan jawaban dibandingkan untuk memverifikasi identitas siswa. Metode yang relatif sederhana ini dapat digunakan untuk mengautentikasi peserta tes; namun, itu tidak dapat digunakan untuk memantau perilaku siswa selama ujian. Selain itu, siswa mungkin masih dapat melewati proses otentikasi dengan memberikan jawaban kepada orang lain agar orang lain mengikuti ujian, atau berkolaborasi dengan orang lain saat mengerjakan ujian. Jadi, jika dipilih, metode ini harus digunakan bersama dengan metode keamanan ujian lainnya untuk memastikan kejujuran akademis.
