\section{Integritas Akademik}
\label{sec:integritasakademik}

Integritas akademik didefinisikan sebagai komitmen terhadap nilai-nilai moral mendasar seperti kejujuran, kepercayaan, keadilan, kesopanan, rasa hormat, dan tanggung jawab \citet{keohane1999fundamental}. Integritas akademik didefinisikan sebagai komitmen terhadap enam nilai inti, yaitu kejujuran, kepercayaan, keadilan, rasa hormat, tanggung jawab, dan keberanian, dalam semua aspek praktik ilmiah, bahkan dalam menghadapi kesulitan. Nilai-nilai ini penting di lembaga pendidikan tinggi untuk evaluasi pembelajaran, tetapi juga karena lembaga ini diharapkan memungkinkan dan mendorong perolehan pengetahuan, pembelajaran individu, pengembangan kemampuan intelektual, pengembangan otonomi dan pemeliharaan reputasi keunggulan akademik sekolah \citep{ahmed2018student} \citep{nuss1984academic}, serta menghasilkan lulusan yang berkontribusi bagi pembangunan ekonomi, sosial dan kemanusiaan negara \citep{muhammad2020factors}, dan yang berperilaku bermoral dalam masyarakat. Dengan demikian, tujuan utama belajar mengajar seperti yang dirasakan saat ini adalah untuk menumbuhkan lingkungan yang berorientasi pada pembelajaran, berdasarkan motivasi pribadi, lebih dari menciptakan lingkungan yang berorientasi pada prestasi \citep{bertram2017academic}. Dan memang, ketika siswa belajar melalui motivasi intrinsik, praktik akademik biasanya adil \citep{barbaranelli2018machiavellian} \citep{krou2020achievement}. Nilai-nilai ini, yang mendasari integritas akademik, dianggap penting bahkan di saat-saat tertekan, yang mungkin berasal dari kurangnya pengetahuan dan ketakutan akan kegagalan \citep{keohane1999fundamental}, atau dari krisis seperti pandemi saat ini \citep{moralista2020faculty}. 

Studi tentang integritas akademik di Spanyol, cakupan geografis makalah ini, tidak memiliki tradisi yang mapan seperti lingkungan Anglo-Saxon atau Eropa tengah dan utara \citep{forgas2009ciberplagio}. Mengenai studi yang berfokus pada perilaku siswa yang tidak jujur terlihat pada saat mereka mengikuti ujian, pekerjaan yang dapat dikutip sangat langka.

Pada sebuah studi yang difokuskan pada mahasiswa keperawatan \citep{blanch2006nivel} menunjukkan bahwa 28\% siswa mengaku telah menyalin selama ujian. Data dari studi kedua, berdasarkan sampel mahasiswa universitas Spanyol, menunjukkan bahwa sekitar 45\% mahasiswa mengaku telah menggunakan lembar contekan dan materi yang tidak diperbolehkan selama ujian \citep{sureda2009practicas}. Sebuah studi kemudian dilakukan oleh kelompok peneliti yang sama, menemukan bahwa hampir 50\% mahasiswa melaporkan telah menyalin setidaknya sekali selama ujian tatap muka \citep{comas2011integridad}. Sebuah panel ahli Spanyol, menyatakan bahwa perilaku tidak jujur paling serius yang dapat dilakukan mahasiswa dalam penilaian mereka adalah: meniru identitas orang lain dalam penilaian; mencuri tes atau ujian, memanipulasi nilai mereka dan mengubahnya untuk orang lain; memperoleh soal ujian atau penilaian sebelum mengikuti ujian; menyerahkan ujian yang diambil oleh siswa lain sebagai ujiannya sendiri; menyontek pada ujian tatap muka melalui perangkat teknologi, seperti ponsel dan earpiece, kemudian, akhirnya, menghadirkan karya siswa lain sebagai miliknya \citep{sureda2020conductas}. Hubungan dan penilaian perilaku tidak jujur menunjukkan kekhawatiran tentang penipuan ujian oleh kelompok ahli yang berpartisipasi.

Adaptasi sistem pendidikan universitas Spanyol dengan konteks yang disebabkan oleh pandemi telah menyebabkan, antara lain, peningkatan kekhawatiran tentang penipuan ujian. Kekhawatiran tersebut telah menghasilkan pengembangan pedoman dan rekomendasi oleh otoritas politik dan akademik tentang prosedur penilaian non-tatap muka \citep{castells2020recomendaciones} \citep{cordon2020informe} \citep{gonzalez2020informe}. Dalam pedoman Konferensi Rektor Universitas Spanyol (Conferencia de Rectores de las Universidades Españolas - CRUE), tidak ada referensi eksplisit untuk menyontek dalam ujian, tetapi hingga dua puluh kali, kata "keamanan" muncul, dengan kejujuran menjadi salah satu dimensi fundamentalnya. Secara khusus, berikut ini dinyatakan:
Aspek penting lainnya yang perlu dipertimbangkan adalah langkah-langkah untuk menjaga integritas akademik dan penggunaan mekanisme hukum yang tersedia (pengusiran dari tes, kualifikasi penangguhan atau, jika perlu, lembaga proses disipliner) dalam kasus tes atau tugas yang curang \citep{cordon2020informe}.

Dalam sebuah buku pegangan yang disiapkan oleh Ministerio de Universidades (Kementerian Universitas), satu bagian didedikasikan untuk menyajikan rekomendasi untuk menghindari penggunaan cara-cara curang dan satu bagian lagi untuk menyajikan sistem untuk menjamin kepenulisan ujian \citet{gonzalez2020informe}.

Buku pegangan rekomendasi yang dikembangkan oleh Kelompok Otoritas Pengajaran Daring Universitas Negeri Castilla y León patut dicatat \citep{garciaevaluacion}. Di antara rekomendasinya adalah mendeteksi peniruan identitas selama ujian sebagai persyaratan yang dapat diminta dari sistem e-proctoring, memblokir browser peserta ujian sehingga mereka tidak dapat mengakses konten di luar ujian, mendeteksi elemen selain yang diperlukan untuk melakukan pengujian; dan, terakhir, mendorong diperolehnya bukti objektif tentang penyelesaian ujian oleh siswa tanpa bantuan atau kerjasama dari pihak ketiga.

Kekhawatiran tentang isu kecurangan ujian online dalam konteks COVID-19 juga tercermin di media-media di Spanyol, yang menggemakan banyak kasus kecurangan dalam penilaian online selama tahun 2020. Di sebagian besar artikel jurnalistik ini, penilaian online disajikan dengan sudut pandang negatif karena potensi kemudahan penipuan. Menurut pendapat Goberna, Profesor Matematika di Universitas Alicante, “Ujian online adalah penipuan; mereka pada dasarnya akan menipu”. Seorang Profesor Filologi Italia di Universitas Oviedo, de Sande, mempertahankan posisi serupa yang menyatakan "Dengan ujian telematika, Anda memberikan kursus".

Indikator akhir sejauh mana fenomena kecurangan penilaian dalam konteks COVID-19 dapat diperoleh dengan menelusuri YouTube dengan deskriptor “copiar examen online” (kecurangan ujian online”). Banyak ditemukan video di mana pengalaman menyontek saat ujian terkait dengan judul langsung, seperti “Ayudo a mi hermana a copiar en un examen online!” (“Saya membantu adik saya menyontek dalam ujian online!”), yang memperoleh lebih dari 3,7 juta tampilan dalam waktu kurang dari sembilan bulan. dan lainnya secara terbuka memberikan saran tentang menyontek pada penilaian online, seperti "Cómo saber las respuestas de un examen online" ("Cara mengetahui jawaban ujian online"), yang mengumpulkan hampir 850.000 tampilan dari April 2020 hingga Februari 2021.

Mempertimbangkan hal-hal yang sudah dirincikan di atas, relevansi memperoleh pengetahuan baru tentang kecurangan ujian di era COVID-19 menjadi jelas. Studi ini membahas masalah dari perspektif yang jarang digunakan sampai sekarang, yaitu analisis data dari pencarian Internet atau analisis pencarian.