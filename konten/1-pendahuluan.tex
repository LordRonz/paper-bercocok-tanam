% Ubah judul dan label berikut sesuai dengan yang diinginkan.
\section{Pendahuluan}
\label{sec:pendahuluan}

% Ubah paragraf-paragraf pada bagian ini sesuai dengan yang diinginkan.

Di bidang teknik elektro, sebuah switch adalah komponen elektronik yang dapat menghubungkan maupun memutuskan jalur konduksi dari sebuah rangkaian listrik, memutus arus listrik atau mengalihkannya dari satu konduktor ke konduktor lainnya. Sederhananya, sebuah switch dapat membuat atau memutuskan rangkaian listrik. Setiap aplikasi listrik dan elektronik menggunakan setidaknya satu switch atau sakelar untuk melakukan operasi ON dan OFF perangkat.

Maka dapat diketahui bahwa switch merupakan bagian dari sistem kontrol, dan tanpa switch, kita tidak dapat memiliki sistem kontrol. Sebuah switch dapat melakukan dua fungsi, yakni ON dengan menutup kontaknya, ataupun OFF dengan membuka kontaknya.

Tipe paling umum dari sebuah switch adalah sebuah perangkat elektromekanikal yang terdiri dari satu atau lebih set kontak listrik yang bergerak yang terhubung pada rangkaian eksternal. Ketika sepasang kontak bersentuhan, arus dapat mengalir di antara kontak tersebut, sedangkan ketika kontak dipisahkan tidak ada arus yang dapat mengalir.
