% Ubah judul dan label berikut sesuai dengan yang diinginkan.
\section{Kesimpulan}
\label{sec:kesimpulan}

% Ubah paragraf-paragraf pada bagian ini sesuai dengan yang diinginkan.

Makalah ini dimulai dengan memberikan tinjauan pemikiran terkini mengenai alasan mengapa siswa mungkin merasa termotivasi untuk terlibat dalam perilaku yang melanggar integritas akademik. Kami mendekati pertanyaan ini dengan mempertimbangkan empat "tingkat" untuk mempertimbangkan integritas akademik: siswa, institusi, media penyampaian, dan penilaian. Kami menyarankan bahwa ketika memeriksa integritas akademik di lingkungan online, akan diperlukan penelitian lanjutan yang mengeksplorasi budaya menyontek dan sifat, dan motivasi untuk, menyontek pada berbagai jenis penilaian. Lebih lanjut, seperti yang ditunjukkan, penelitian hingga saat ini telah menghasilkan temuan yang beragam dalam kaitannya dengan apakah ketidakjujuran akademik mungkin lebih atau kurang lazim di lingkungan online, dan kami telah menyerukan penelitian lebih lanjut yang meneliti jenis penilaian, bidang studi, dan demografi siswa (mis. , usia dan alasan mendaftar di kursus). Di paruh kedua ulasan ini, kami merinci metode untuk mencegah dan mendeteksi perilaku menyontek, dengan fokus pada penilaian sumatif online. Kami menekankan lagi, bahwa metode ini harus dipertimbangkan bersama dengan pertimbangan yang lebih luas dari alasan dan motivasi bahwa siswa mungkin terlibat dalam ketidakjujuran akademik di tempat pertama, dan dengan perhatian eksplisit dan perhatian terhadap privasi siswa dan perlakuan yang adil.

Tantangan atau pertanyaan keamanan adalah salah satu metode paling sederhana untuk mengautentikasi peserta tes. Metode ini membutuhkan pengetahuan pribadi untuk mengotentikasi siswa dan disebut sebagai metode 'otentikasi berbasis pengetahuan'. Siswa diberi pertanyaan pilihan ganda berdasarkan sejarah pribadi mereka, seperti informasi tentang alamat rumah masa lalu mereka, nama sekolah menengah mereka, atau nama gadis ibu mereka. Siswa harus menjawab pertanyaan-pertanyaan ini untuk mengakses ujian, dan pertanyaan juga dapat ditanyakan secara acak selama penilaian .

Integritas akademik tetap menjadi elemen integral dari pendidikan tinggi. Nilai-nilai prinsip yang membentuk integritas akademik tidak hanya menjunjung tinggi reputasi universitas dan nilai serta makna gelar yang dianugerahkannya, tetapi juga menciptakan kerangka kerja bersama untuk pekerjaan profesional yang diperluas di luar akademi. Oleh karena itu, karena studi online terus berkembang dalam pendidikan pasca sekolah menengah, kami percaya bahwa penting untuk mengembangkan beasiswa dan diskusi mengenai pemeliharaan integritas akademik di lingkungan online.
