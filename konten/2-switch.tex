\section{Switch}
\label{sec:switch}

Switch dibuat dari banyak konfigurasi yang berbeda. Switch-switch ini mungkin dapat memiliki beberapa set kontak yang dikendalikan dengan menggunakan knob atau aktuator yang sama, dan kontak ini dapat bekerja secara bersamaan, berurutan, ataupun berselang-seling. Sebuah switch mungkin saja bekerja secara manual. Misalnya saja sebuah switch untuk lampu ataupun tombol di keyboard. Switch juga mungkin saja bekerja sebagai elemen sensor untuk mendeteksi posisi, level dari cairan, tekanan, maupun temperatur, misalnya saja thermostat. Bentuk-bentuk lainnya.

Switch ini juga memiliki berbagai macam bentuk, seperti toggle switch, rotary switch, mercury switch, push-button switch, reversing switch, relay, dan circuit breaker. Penggunaan dari switch yang umum adalah kontrol pencahayaan, di mana beberapa switch dapat dihubungkan ke satu rangkaian untuk memungkinkan kontrol yang mudah dari perlengkapan lampu. Switch yang terintegrasi dengan rangkaian yang berdaya tinggi harus memiliki konstruksi khusus untuk mencegah busur api yang dapat merusak saat dibuka.
